\Chapter{MinPath: an interactive optimization program}

Now we assume that  the files `/tmp/eight.sym' and `/tmp/eight.info' exist. 
We can interactively perform and visualize the minimization process
as follows.

\Section{Main commands}

\> x=minpath(<symfile>)

Define a MinPath object with symmetry group stored in  the 
file <symfile>. 
If no <symfile> is present, a list of symfiles in the current
directory is given.
Otherwise, there are the following pre-built objects:
`eight_c6', 
'eight_d6',
'eight_d12',
'trivial',
'line',
'isosceles',
'hill',
'choreography',
'lagrange',
'choreography_21' (from the list in \cite{zz}).


\> x.info()

Show info file of the path object `x'.

\> x.new(*SPMETHOD=<SPMETHOD>)

Give coefficients to `x'. If SPMETHOD=<SPMETHOD> is present,
use StartingPath Method <SPMETHOD>. For example, SPMETHOD=1
assigns random coefficients to the path.



\> x.relax() 

Optimization subroutines. It gives the list of all available algorithms
with a short description and their codes, as follows:
\begintt
0:      Unconstrained Minimization with analytic Gradient
1:      Unconstrained Minimization with Analytic Hessian
2:      Unconstrained Minimization with finite-Difference 
Hessian
3:      Unconstrained Minimization with Conjugate 
Gradient and analytic Gradient
4:      Unconstrained Minimization without gradient 
(nonsmooth)
5:      Linearly Constrained Minimization with Analytic 
Gradient
6:      Step-Flow Descent
7:      Simple Conjugate Gradient
------------------------------------------------------------------------
100:    Newton-Powell Finite-difference Jacobian
200:    Newton-Powell Analytic Jacobian
300:    Secant Broyden's Update and Finite-difference 
Jacobian
400:    Secant Broyden's Update and Analytic Jacobian
\endtt


\> x.relax(<relax\_code> + <newton\_code>)

Optimization:  perform an optimization algorithm number <relax_code>
and subsequently a newton root-finding with algorithm <newton_code>.

\> x.newton(<newton_code>)

Optimization:  perform
a newton root-finding with algorithm <newton_code>.



\> x.view() 

View the path (GEOMVIEW has to be installed).

% \) {\fmark x.omega=[<omegax>,<omegay>,<omegaz>]}  

Set the rotation vector $\omega$.

\> x.reshape(<STEPS>)

Reshape the coefficient matrix of `x'.


\> x.load(<filename>)

Load coefficients for `x' from <filename>.

\> x.write(<filename>)

Write coefficients for `x' to <filename>.

\> x.printsol(<filename>)

Write to the file <filename> the positions of the <NOB> bodies
in time, in a gnuplot-like format. It is the format that 
can be visualized by `orbview'.

\> x.dump()

For debug only. Write to stdout everything.

\> x.action()

Compute the action of the loop `x'.

\> x.howsol()

Compute the norm of the gradient of the action of `x'.


\> x.withcoll

Set this variable to 1 if we want to use minorb avoiding collisions,
or 0 otherwise.

\Section{A database of planar symmetry groups}

\) {\fmark all\_minpaths(dim=<dim>,NOB=<NOB>,GroupOrder=<group_order>,KernelTauOrder=<kernel_tau_order>,ActionType=<action_type>,IsTypeR=<IsTypeR>,IsCoercive=<IsCoercive>,IsRedundant=<IsRedundant>,HasAlwaysCollisions=<HasAlwaysCollisions>,TDSizes=<TDSizes>)}

All options are not necessary. It gives back the list of all
minpath objects with the desired properties (all of them have to 
be fulfilled).

\begintt
MinorbShell > len(all_minpaths(NOB=3,dim=2,IsCoercive=true,\
IsRedundant=false,HasAlwaysCollisions=false))
6
MinorbShell
\endtt


\Section{An example session}
In the following example we load the lagrange symmetry,
minimize with a Conjugate Gradient, then 
solve the gradient system around the minimum with a Newton-Powell
method, with analytic jacobian. After this, we reshape
the solution, doubling its coefficients, and re-do the 
nonlinear rootfinding with Broyden's Update and analytic Jacobian.
The solution is written in the file `/tmp/lag1.myg'.

First, it is necessary to run the interpreter. It is a subshell
of `python', so that all python syntax and modules are 
accessible insider `minpath'.

\begintt
[unix_shell] $ minpath
minpath -- beginning at Fri Mar 26 14:37:51 CET 2004

symfiles:


MinorbShell >
\endtt

Now we can use the interactive shell.

\begintt
MinorbShell > x=lagrange
MinorbShell > x.new()
 starting new path...
MinorbShell > x.relax(203)
 relaxing...
 # using IMSL DUMCGG
 # Unconstrained Minimization with Conjugate Gradient and 
analytic Gradient
 # using NONLINEAR DNEQNJ
 # Newton-Powell Analytic Jacobian
 ==> action:   4.7124; howsol: 1.5637e-15
MinorbShell > x.view()
MinorbShell > x.reshape(50)
<minpath object; NOB=3, dim=2, steps=50>
MinorbShell > x.newton(400)
 computing newton path...
 # using NONLINEAR DNEQBJ
 # Secant Broyden's Update and Analytic Jacobian
 ==> action:   4.7124; howsol: 1.5688e-15
MinorbShell > x.write('/tmp/lag1.myg')
\endtt

In the following example we consider the symmetry file that
we have built in the previous chapter, and compute its
minimizer with finite-difference Hessian and Newton-Powell
with analytic Jacobian.

\begintt
MinorbShell > ei=minpath('/tmp/eight.sym')
MinorbShell > ei.info()
info on /tmp/eight:
----------------------------------------------------------------
@ GroupOrder: 12
@ KernelTauOrder: 1
@ ActionType: 2
@ IsTypeR: false
@ IsCoercive: true
@ IsRedundant: false
@ HasAlwaysCollisions: false
@ TransitiveDecomposition: [ [ 1, 2, 3 ] ]
@ TypeRDirections: [  ]
x_1(t+T/6) = [ [ -1.0, 0.0 ], [ 0.0, 1.0 ] ] * x_2(t)
x_2(t+T/6) = [ [ -1.0, 0.0 ], [ 0.0, 1.0 ] ] * x_3(t)
x_3(t+T/6) = [ [ -1.0, 0.0 ], [ 0.0, 1.0 ] ] * x_1(t)
x_1(-t) = [ [ -1.0, 0.0 ], [ 0.0, -1.0 ] ] * x_2(t)
x_2(-t) = [ [ -1.0, 0.0 ], [ 0.0, -1.0 ] ] * x_1(t)
x_3(-t) = [ [ -1.0, 0.0 ], [ 0.0, -1.0 ] ] * x_3(t)
----------------------------------------------------------------

MinorbShell > ei.new()
 starting new path...
MinorbShell > ei.relax(202)
 relaxing...
 # using IMSL DUMIDH
 # Unconstrained Minimization with finite-Difference 
Hessian
 # using NONLINEAR DNEQNJ
 # Newton-Powell Analytic Jacobian
 ==> action:   3.6906; howsol: 9.3383e-16
MinorbShell > ei.view()
MinorbShell > ei.write('/tmp/eight.myg')
\endtt

\Section{Parallel cluster}

% After installing lamboot.py and parminpath in all the nodes of the cluster,
% under a no-password access of the cluster site, simply run:

Again, make password-less access on a grid (SGE) available. 
Export the variable {\tt REMOTE_MINORB=hostname},
 then {\tt minpath} will (probably) be able to connect remotely.
Then the nice function {\tt remjob} will be available:


\begintt
MinorbShell > res=remjob(x,200,"new();relax(2);newton(300)" )
\endtt

for example (the syntax is obvious).


The environment variable {\tt SYMORBDIR} has to be set on the nodes 
to locate the proper directory.
\begintt
. /share/SGE6.1/default/common/settings.sh
export LD_LIBRARY_PATH=${LD_LIBRARY_PATH}:/usr/local/lf9562/lib
export PATH=${PATH}:${HOME}/local/bin

\endtt


Other useful pieces of code:
\begintt
rr=filter(lambda y: y.howsol() < 0.001, res)
rr2=[x.newton(200) for x in rr]
\endtt
