\Chapter{Symorb Manual}

The main object of the package is the 
Lagrange symmetry group (in short LGS). 
In the first section we show the usage of the main commands,
then some helpful utilities, and then 
the methods, properties and attributes of LGS's. 

\Section{Main commands}

\> LagSymmetryGroup(<action_type>,<NOB>, <kern>, <rotV>, <rotS>, <refV>, <refS>)

The arguments are: <action_type> is the action type
(restricted: 0 if cyclic and anything else otherwise),
<NOB> is the number of bodies,
<kern> is the kernel of $\tau$, <rotV> is 
the matrix of the cyclic generator acting on the Euclidean space,
<rotS> is the permutation corresponding to the cyclic generator,
<refV> is the reflection in the space
and <refS> is the reflection in the index set (if the action
type is cyclic they are ignored).

\> TrivialKerTau( <dim> )

Returns a trivial kernel of $\tau$ for a space of dimension
<dim>; it can be used with `LagSymmetryGroup'.


\> MinorbInitString(<LSG>)

Build the init string. <LSG> is a symmetry group, as defined above.
The init string is the parseable string necessary to pipe 
to the minimization engine.

\> MinorbInfoString(<LSG>)

Build the info string. <LSG> is a symmetry group, as defined above.
The info string is a human-readable string that describes
the LSG and some of its properties.

\> MakeMinorbSymFile(<basefilename>,<LSG>)

Create `<basefilename>.init' and `<basefilename>.info',
where it is stored the init symfile and the info file, produced
by "MinorbInitString" and "MinorbInfoString".



\> AllLSGTK (<action_type>,<NOB>,<dim>)

It returns a list of all LGS's with trivial kernel of $\tau$,
action type equal to `action_type', `NOB' bodies and dimension
`dim'.


\Section{Utilities, properties and attributes}

\> RotationMatrixDim2(<order>)

A rotation `2x2' matrix of order <order>.

\> RotationMatrixDim3(<order>)

A rotation `3x3' matrix of order <order>. It fixed the third
coordinate.

\> IsTypeRDirection (<LSG>,<dir>)

It retuns true or false according to whether the direction
<dir> (in $1$, $2$, or $3$) is a direction under which 
the symmetry group is of type R.

\> IsTypeR (<LSG>)

True if <LSG> is of type R.

\> TransitiveDecomposition(<LSG>)

It yields the transitive decomposition of the action of the group
on the index set.

\> IsTransitiveLSG(<LSG>)

True if the group is transitive on the index set.

\> IsValidLSG(<LSG>)

It checks whether the <LSG> is well-defined or not.

\> IsCoercive(<LSG>)

True if <LSG> is coercive.

\> HasAlwaysCollisions(<LSG>)

If true, then  <LSG> has always collisions.  If false, mostly
it has not. (To be improved).


\> ActionType(<LSG>)

It returns $0$ if the action type is cyclic, $1$ if it is brake
and $2$ if it is dihedral.

\> IsRedundant(<LSG>)

True if <LSG> is redundant.

\> GroupOrder(<LSG>)

The order of the group.

\> KernelTauOrder(<LSG>)

The order of the kernel of $\tau$.

\Section{An Example Session}

\begintt
gap> RequirePackage("symorb");
true
gap> NOB:=3;dim:=2;
3
2
gap> rotV:=[[-1,0],[0,1]];
[ [ -1, 0 ], [ 0, 1 ] ]
gap> rotS:=(1,2,3);
(1,2,3)
gap> refV:=[[-1,0],[0,-1]];
[ [ -1, 0 ], [ 0, -1 ] ]
gap> refS:=(1,2);
(1,2)
gap> LSG:= LagSymmetryGroup(1,NOB, TrivialKerTau(2), rotV, rotS, refV, refS);
LagSymmetryGroup(NOB=3, dim=2, action_type=2, rotation=Tuple( [
[ [ -1, 0 ], [ 0, 1 ] ], (1,2,3) ] ), reflection=Tuple( [
[ [ -1, 0 ], [ 0, -1 ] ], (1,2) ] )
gap> s:=MinorbInfoString(LSG);;
gap> Print(s);
% SYMORB version : 4r2 fix8-0.9
% date        : Fri Mar 26 13:53:48 CET 2004
% on          : Linux i586 unknown
@ GroupOrder: 12
@ KernelTauOrder: 1
@ ActionType: 2
@ IsTypeR: false
@ IsCoercive: true
@ IsRedundant: false
@ HasAlwaysCollisions: false
@ TransitiveDecomposition: [ [ 1, 2, 3 ] ]
@ TypeRDirections: [  ]
x_1(t+T/6) = [ [ -1.0, 0.0 ], [ 0.0, 1.0 ] ] * x_2(t)
x_2(t+T/6) = [ [ -1.0, 0.0 ], [ 0.0, 1.0 ] ] * x_3(t)
x_3(t+T/6) = [ [ -1.0, 0.0 ], [ 0.0, 1.0 ] ] * x_1(t)
x_1(-t) = [ [ -1.0, 0.0 ], [ 0.0, -1.0 ] ] * x_2(t)
x_2(-t) = [ [ -1.0, 0.0 ], [ 0.0, -1.0 ] ] * x_1(t)
x_3(-t) = [ [ -1.0, 0.0 ], [ 0.0, -1.0 ] ] * x_3(t)
gap> MakeMinorbSymFile("/tmp/eight",LSG);
file /tmp/eight.sym created!
file /tmp/eight.info created!
0
gap> IsTransitiveLSG(LSG);
true
gap> IsCoercive(LSG);
true
gap> HasAlwaysCollisions(LSG);
false
\endtt

\Section{Example of generating file}

\begintt
RequirePackage("symorb");
NOB:=12;
dim:=3;
mat1:=[[-1,0,0],[0,-1,0],[0,0,1]];
mat2:=[[0,1,0],[0,0,1],[1,0,0]];
G:=GroupWithGenerators([mat1,mat2]);
hom:=ActionHomomorphism(G,G,OnRight);
s1:=Image(hom,mat1);
s2:=Image(hom,mat2);
kert:=GroupWithGenerators([ Tuple([mat1,s1]), Tuple([mat2,s2]) ] );
rotV:=[[-1,0,0],[0,-1,0],[0,0,-1]];
rotS:=();

LSG:=LagSymmetryGroup(0,NOB,kert, rotV,rotS,rotV,rotS);
MakeMinorbSymFile("try",LSG);
\endtt


\Section{Generating LSGs}

\> LagSymmetryGroupCHARS(<NOB>, <Tchar>, <Vchar>, <sigma>)
Constructor for the new object with rec...


\> MakeLSGfromCHARS(<stru>);
Where <stru> is build with |LagSymmetryGroup|


\> MakeActions(<maxorbs>,<group>,<n>,<dim>)








